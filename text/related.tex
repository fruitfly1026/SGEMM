\section{Related Work}
\label{sec:related}
%To our knowledge, this paper provides the first comprehensive study of demystifying NVIDIA GPU microarchitecture which 
%is correlated with performance tuning in SGEMM. 
This section briefly discusses the related work in reverse engineering ISA encoding, GPU assembler, microbenchmarking, and SGEMM optimization.

{\em {\bf ISA Encoding Solver and GPU Assembler}}:
CPU instruction reverse engineering work~\cite{collberg1997reverse,hsieh2001reverse} broadly exists by using a MIPS, SPARC, Alpha, PowerPC, ARM, or x86 assembler as a tool to crack CPU instruction sets and extract bit-level instruction encoding information. 
These work outputs C declarations that can be used by binary tools. 
However, NVIDIA does not have an available GPU assembler, but a disassembler instead. 
The lack of a GPU assembler for public use makes our reverse engineering on GPUs more meaningful. 


The absence of a GPU assembler motivates a series of work on developing toolchains to facilitate tuning codes at assembly level. 
For the early architecture G80, Decuda~\cite{decuda} demonstrated the feasibility to operate
assembly instructions. 
After that, for almost every generation of GPU architecture, there are several 
efforts on developing assembly toolchains. 
Asfermi~\cite{asfermi} and cudaasm-qhasm~\cite{bernstein2012usable} are assemblers for Fermi architecture, and MaxAs~\cite{maxas} assembler is for Maxwell architecture. 
% by reverse engineering the encoding of Maxwell GPU. 
Envytools~\cite{envytools} supports translation of PTX instruction to $64$-bit binaries 
for several GPU architectures, it is not able to generate a compatible {\tt cubin} which can be directly used by the CUDA driver APIs. %Besides, no prior work presents the detailed instruction decoding algorithms process as ours.
% It's no doubt that 
Although all their work involves reverse engineering GPU ISA encodings, neither detailed descriptions are given nor a general tool is developed to crack a new GPU ISA encoding. 
%Hsieh use CPU assemblers to generate ISA encoding of CPU platforms~\cite{collberg1997reverse,engler2000derive} to be used by binary tools .
We provide a general toolchain to automatically crack different GPU ISA encodings and a complete assembler for Kepler architecture.
%A remarkable advantage of our assembly toolchain is the compatibility with CUDA's {\tt cuobjdump}. 

%We share the same idea of performance microbenchmarking with other works. 
{\em {\bf Benchmarking}}: 
Fog~\cite{fog} did a sound work on benchmarking instruction latency and throughput of several generations of Intel and AMD CPUs and Intel Xeon Phi accelerators. %The benchmarks focused on instruction latency, throughput and memory.
Wong et al.~\cite{wong} performed a 
comprehensive benchmarking work on GT200 and provided pipeline latency data and
memory features. Mei et al.~\cite{mei} benchmarked memory hierarchy including cache, shared memory on Fermi, Kepler and Maxwell GPUs.
However, they haven't benchmarked dual-issue mode of arithmetic instructions or vectorized load instructions such as {\tt LD} and {\tt LDS}, which leaves non-negligible unexplored performance space. 
%Due to lack of considering vectorized load instructions and 
%too less instructions inside the loop, the results is not so accurate.
% This leads to results with unsatisfying accuracy.
% No vectorized load instructions are benchmarked and too few instructions inside the loop 
We leverage the complete assembler to crack control codes which reveal more microarchitecture details for tuning application performance.

{\em {\bf Matrix Multiplication Tuning}}: 
With respect to GEMM optimization in microarchitectural level, some work inspires our GPU implementations. 
Tan et al.~\cite{tan} implemented a fast DGEMM by using 
assembly level optimization, such as software pipelining, vector memory operations, and instruction scheduling. 
Lai et al.~\cite{lai} presented performance analysis and optimization work of SGEMM on both Fermi and GTX680 GPUs. 
%\jli{deleted, because of useless: ``However, 
%they didn't consider dual-issue mode by setting control codes, which can boost performance significantly.'' }
Gray~\cite{nervana_sgemm_wiki} presented SGEMM optimization in assembly on a Maxwell GPU. 
%\jli{deleted, useless: ``Since Maxwell does not support {\tt FFMA} dual issue, the optimization is not as complex as Kepler.'' }
We adopt the proved effective optimization techniques such as shared memory/register 
blocking and double-buffering~\cite{volkov,tan}.
However, it is difficult to tune all possible microarchitectural optimizations well without a useful toolchain.
In this work, we present a more complete case study of applying microarchitectural features by combining 
register allocation, dual issuing, memory access paths, and instruction scheduling by using our toolchains.
