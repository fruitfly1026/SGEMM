\begin{abstract}
In this paper, we present a methodology to demystify GPU's microarchitectural level features and demonstrate how to use
them to tune SGEMM. The methodology relies on a reverse engineering approach to crack the GPU ISA instruction encoding in
order to build a GPU assembler, and a profound microbenchmarking at assembly level to correlate architecture features with
performance factors.
We uncover microarchitectural features, such as float instruction dual issue, memory load width,
and register bank. 
Based on these features, we implement the fastest SGEMM on Kepler GPU.
The performance boost is achieved on the basis of tuning {\tt FFMA} throughput to the
hardware peak by activating dual issue and eliminating bank conflicts, then adding other non-FFMA instructions with little
penalty by choosing proper width of global/shared load instructions, and instruction scheduling.
The optimized SGEMM achieves $3104$ Gflop/s and its efficiency is $88\%$, which is 12\% higher than cuBLAS7.0 on NVIDIA Kepler K20 GPU.
We believe that our SGEMM optimization can serve as an example of how to optimize other float computation-intensive algorithm on GPU.
\end{abstract}

