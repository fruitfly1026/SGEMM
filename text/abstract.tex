\begin{abstract}
In this paper, we present a methodology to demystify GPU microarchitectural features and demonstrate the performance tuning experience for compute-intensive kernels. The methodology relies on a reverse engineering approach to cracking the GPU ISA instruction encodings to build a GPU assembler.
An assembly microbenchmark suite correlates microarchitectural features with their performance factors and uncovers float instruction dual issue, memory load width, register bank, and other microarchitectural features. We use SGEMM as a running example throughout this paper to show how to achieve bare-metal performance tuning. The performance boost is achieved by tuning {\tt FFMA} throughput to the hardware peak by activating dual issue, eliminating register bank conflicts, adding non-{\tt FFMA} instructions with little penalty, and choosing proper width of global/shared load instructions.
On NVIDIA Kepler K20m, we develop a faster SGEMM with $3104$Gflop/s performance and $88\%$ efficiency, which is $15\%$ higher than cuBLAS7.0 .
Applying these optimizations on convolutional networks, the optimized implementations gains $39\%$-$62\%$ performance improvement compared to cuDNN4.0.
The toolchain is an attempt to automatically crack different GPU ISA encodings and build a complete assembler adaptively, to explore more performance for applications on GPUs.
% , and instruction scheduling.
% The optimized SGEMM achieves $3104$ Gflop/s and its efficiency is $88\%$, which is $15\%$ higher than cuBLAS7.0 on NVIDIA Kepler K20m.
% We believe that our SGEMM optimization can serve as an example of how to optimize other float computation-intensive algorithm on GPU.
\end{abstract}

