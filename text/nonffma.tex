\subsection{Schedule non-FFMA Instructions}
After setting the order of {\tt FFMA}, other {\tt non-FFMA} instructions should be inserted in proper positions to
assure the correct program without losing performance. In order to tolerate instruction latency, the
distance of dependent instructions needs to be larger than their latency. The distance is approximated as
\begin{equation}
\label{eq:inst}
distance = \frac{4\times\#instructions}{7}.
\end{equation}
A $7$-instruction scheduling block costs $4$ clock cycles to be issued in dual issue mode.
Therefore, if we assume two interleave instructions have a distance $L$, at least $\frac{L*7}{4}$ instructions are needed.
% two instructions of $L$ distance, then $\frac{L*7}{4}$ instructions are needed.
Besides, the remained number of slots
to insert these instructions is estimated as

\begin{displaymath}
\#slots = \frac{rx\times ry\times bk}{ffmas\_in\_schedule\_block}=\frac{12\times 12\times 4}{6}=24\times 4.
\end{displaymath}
$rx\times ry\times bk$ yields the total number of {\tt FFMA}s for one thread inside register blocking loop in Algorithm~\ref{gemm}, where $rx$ and $ry$ are register blocking size and $bk$ is the unrolling factor.
$ffmas\_in\_schedule\_block$ is the number of {\tt FFMA} instructions inside one scheduling block, which is $6$ by our
$1$-$2$-$2$-$1$ dual issue pattern in section~\ref{sec:benchmark}.
According to these principles, we first arrange {\tt LDS}, {\tt STS}, {\tt LDG} because of their long latencies. The
schedule slots are illustrated in two dimension in Table~\ref{tab:position}.
Note that we use double buffers to hide the latency of {\tt LDG} from global memory, which is $200$ clock cycles.
Every $4$ loops require $2$ {\tt LDG}s to load data from global memory to registers, $4$ {\tt STS}s to store data from
registers to shared memory. A read after write (RAW) dependency exists between {\tt
LDG} and {\tt STS}.
From equation~\ref{eq:inst}, $\frac{200\times 7}{4} = 350$ instructions are needed between them.
We put {\tt LDG} and  {\tt STS} in position $P[77][3]$ and $P[65][2]$ respectively in Table~\ref{tab:position}.
Thus, $143-77 + 144\times 2 + 65=419$ ($>350$) instructions are between them, which is enough to hide latency of {\tt LDG}s.
% resulting in a distance of $\frac{4\times 419}{7}=239$ clocks,.

The arrangement of {\tt LDS}s, loading data from shared memory for double buffers of $A$ and $B$, follows the same approach with {\tt LDG}s.
A {\tt LDS} has a latency of $40$ clock cycles, thus $\frac{40\times 7}{4}=70$ instructions are needed to interleave {\tt LDS} and {\tt FFMA}.
In Table~\ref{tab:position}, {\tt LDS} in $P[11][3]$ reads data from {\tt STS} in $P[65][2]$,
the distance between them is more than $40$ clock cycles.
At the end, a {\tt BAR.SYNC} is inserted after {\tt STS} but before {\tt LDS} to make sure that data in shared memory is ready.
Other instructions such as {\tt XOR}, {\tt IADD}, {\tt ISETP} are inserted according to data dependency, they doesn't influence the performance because of their short latencies.

\begin{table}[htbp]
\caption{The position table of {\tt Non-FFMA} instructions. The inner-loop is unrolled by 4 times. The first column
records slot numbers and the first row represents iteration number.}
\label{tab:position}
\captionsetup{font=scriptsize}
% \centerin
\scalebox{0.78} {
\begin{tabular}{|c|c|c|c|c|}
\hline
\diagbox[width=4em, height=3em]{slot}{unroll} & 0 &1 &2 &3 \\
    \hline
    5 & ISET P0 & IADD A0 & & XOR smB \\
    \hline
    11 & LDS.64 smA & LDS.64 smA & LDS.64 smA & LDS.64 smA \\
    \hline
    17 & LDS.64 smA & LDS.64 smA & LDS.64 smA & LDS.64 smA \\
    \hline
    23 & LDS.64 smA & LDS.64 smA & LDS.64 smA & LDS.64 smA \\
    \hline
    29 & LDS.64 smA & LDS.64 smA & LDS.64 smA & LDS.64 smA \\
    \hline
    35& IADD K, -4 & IADD A1 & TEXDEPBAR & \\
    \hline
    41 & LDS.64 smB & LDS.64 smB & LDS.64 smB & LDS.64 smB \\
    \hline
    47 & LDS.64 smB & LDS.64 smB & LDS.64 smB & LDS.64 smB \\
    \hline
    53 & LDS.64 smB & LDS.64 smB & LDS.64 smB & LDS.64 smB \\
    \hline
    59 & LDS.64 smB & LDS.64 smB & LDS.64 smB & LDS.64 smB \\
    \hline
    65 & & &STS.64 writeS & ISETP P2 \\
    \hline
    71 & & & & \\
    \hline
    77 & & IADD B0 & & LDG A \\
    \hline
    83 & LDS.64 smA & LDS.64 smA & LDS.64 smA & LDS.64 smA \\
    \hline
    89 &ISETP P3 & & &\\
    \hline
    95 & LDS.64 smA & LDS.64 smA & LDS.64 smA & LDS.64 smA \\
    \hline
    101 & & & STS.64 loadB0 & LDG B \\
    \hline
    107 & & & STS.64 loadB2 & XOR writeS \\
    \hline
    113 & & & & \\
    \hline
    119 & LDS.64 smB & LDS.64 smB & LDS.64 smB & LDS.64 smB \\
    \hline
    125 & & & XOR smA & \\
    \hline
    131 & LDS.64 smB & LDS.64 smB & LDS.64 smB & LDS.64 smB \\
    \hline
    137 & & & & \\
    \hline
    143 & & IADD B1 & BAR.SYNC & BAR Loop \\
    \hline
\end{tabular}
}
\end{table}
